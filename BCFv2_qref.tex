\documentclass[10pt]{article}
\usepackage{color}
\definecolor{gray}{rgb}{0.7,0.7,0.7}
\usepackage{framed}
\usepackage{enumitem}
\usepackage{longtable}

\addtolength{\textwidth}{3.4cm}
\addtolength{\hoffset}{-1.7cm}
\addtolength{\textheight}{4cm}
\addtolength{\voffset}{-2cm}

\makeindex

\title{BCF2 Quick Reference}
\author{}
\date{}

\begin{document}
\maketitle

{\small
In BCF2, a typed value consists of a typing byte and the actual value with type
mandated by the typing byte. In the typing byte, the lowest four bits (bit
0--3) give the atomic type. The number represented by the higher 4 bits
indicates the size of the following array if it is below 15; if the number
equals 15, the following typed integer gives the array length.
\begin{center}
{\small\begin{tabular}{rlrl}
\hline
Bit 0--3 & C type & Missing value & Description \\
\hline
1 & {\tt int8\_t}   & {\tt 0x80}               & signed 8-bit integer \\
2 & {\tt int16\_t}  & {\tt 0x8000}             & signed 16-bit integer \\
3 & {\tt int32\_t}  & {\tt 0x80000000}         & signed 32-bit integer \\
5 & {\tt float}     & {\tt 0x7F800001}         & IEEE 32-bit floating pointer number \\
7 & {\tt char}      & `{\tt \char92 0}'        & character \\
9 & {\tt char*}     & ``{\tt\char92 0}'' & null-terminated C string \\
\hline
\end{tabular}}
\end{center}
}
\begin{table}[ht]
\centering
{\small
\begin{tabular}{|l|l|l|p{8.2cm}|l|r|}
  \cline{1-6}
  \multicolumn{3}{|c|}{\bf Field} & \multicolumn{1}{c|}{\bf Description} & \multicolumn{1}{c|}{\bf Type} & \multicolumn{1}{c|}{\bf Value} \\\cline{1-6}
  \multicolumn{3}{|l|}{\sf magic} & BCF2 magic string & {\tt char[4]} & {\tt BCF\char92 2}\\\cline{1-6}
  \multicolumn{3}{|l|}{\sf l\_text} & Length of the header text, including any {\sf NULL} padding & {\tt uint32\_t} & \\\cline{1-6}
  \multicolumn{3}{|l|}{\sf text} & {\sf NULL}-terminated plain VCF header text & {\tt char[{\sf l\_text}]} & \\\cline{1-6}
  \multicolumn{6}{|c|}{\textcolor{gray}{\it List of VCF records (until the end of the BGZF section)}} \\\cline{2-6}
  & \multicolumn{2}{l|}{\sf l\_shared} & Data length from {\sf CHROM} to the end of {\sf INFO} & {\tt uint32\_t} & \\\cline{2-6}
  & \multicolumn{2}{l|}{\sf CHROM} & Reference sequence ID & {\tt int32\_t} & \\\cline{2-6}
  & \multicolumn{2}{l|}{\sf POS} & 0-based leftmost coordinate & {\tt int32\_t} & \\\cline{2-6}
  & \multicolumn{2}{l|}{\sf QUAL} & Variant quality; {\tt 0x7F800001} for a missing value & {\tt float} & \\\cline{2-6}
  & \multicolumn{2}{l|}{\sf ID} & {\sf NULL} terminated string; field length inferred & {\tt char*} & \\\cline{2-6}
  & \multicolumn{2}{l|}{\sf REF} & {\sf NULL} terminated string; field length inferred & {\tt char*} & \\\cline{2-6}
  & \multicolumn{2}{l|}{\sf n\_alt} & Number of {\sf ALT} alleles & {\tt uint16\_t} & \\\cline{2-6}
  & \multicolumn{2}{l|}{\sf ALT} & {\sf NULL} terminated string; alleles delimited by {\sf NULL} & {\tt char*} & \\\cline{2-6}
  & \multicolumn{2}{l|}{\sf FILTER} & List of filters; filters are defined in the dictionary & {\tt typed vec} & \\\cline{2-6}
  & \multicolumn{2}{l|}{\sf n\_info} & Number of info key-value pairs & {\tt uint16\_t} & \\\cline{2-6}
  & \multicolumn{5}{c|}{\textcolor{gray}{\it List of key-value pairs in the INFO field (n=n\_info)}} \\\cline{3-6}
  & & {\sf info\_key} & Info key, defined in the dictionary & {\tt typed int} & \\\cline{3-6}
  & & {\sf info\_value} & Value & {\tt typed val} &\\\cline{2-6}
  & \multicolumn{2}{l|}{\sf l\_indiv} & Data length of {\sf FORMAT} and individual genotype fields & {\tt uint32\_t} & \\\cline{2-6}
  & \multicolumn{2}{l|}{\sf n\_fmt} & Number of format fields & {\tt uint16\_t} & \\\cline{2-6}
  & \multicolumn{5}{c|}{\textcolor{gray}{\it List of formats and sample information (n=n\_fmt)}} \\\cline{3-6}
  & & {\sf fmt\_key} & Format key, defined in the dectionary & {\tt typed int} & \\\cline{3-6}
  & & {\sf fmt\_type} & Typing byte of each individual value, possibly followed by a typed int for the vector length & {\tt uint8\_t+} & \\\cline{3-6}
  & & {\sf fmt\_value} & Array of values. The information of each individual is concatenated in the array. Every value is of the same {\sf fmt\_type}.
    Variable-length arrays are padded with missing values; a string is stored as an array of {\tt char}. & (by {\sf fmt\_type}) &\\
  \cline{1-6}
\end{tabular}}
\end{table}


\end{document}
