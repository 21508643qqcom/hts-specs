\documentclass[10pt]{article}
\usepackage{framed}
\usepackage{enumitem}
\usepackage{longtable}

\addtolength{\textwidth}{3.4cm}
\addtolength{\hoffset}{-1.7cm}
\addtolength{\textheight}{4cm}
\addtolength{\voffset}{-2cm}

\makeindex

\title{The SAM Format Specification (v1.3-r624)}
\author{The SAM Format Specification Working Group}
\begin{document}

\maketitle


\section{The SAM Format Specification}
SAM stands for Sequence Alignment/Map format. It is a TAB-delimited text
format consisting of a header section, which is optional, and an
alignment section. If present, the header must be prior to the
alignments. Header lines start with `{\tt @}', while alignment lines do
not. Each alignment line has 11 mandatory fields for essential alignment
information such as mapping position, and variable number of optional
fields for flexible or aligner specific information.

\subsection{An example}
Suppose we have the following alignment with bases in lower cases
clipped from the alignment. Read {\tt r001/1} and {\tt r001/2}
constitute a read pair; {\tt r003} is a chimeric read; {\tt r004}
represents a split alignment.

\begin{framed}\small
\begin{verbatim}
Coor     12345678901234  5678901234567890123456789012345
ref      AGCATGTTAGATAA**GATAGCTGTGCTAGTAGGCAGTCAGCGCCAT

+r001/1        TTAGATAAAGGATA*CTG
+r002         aaaAGATAA*GGATA
+r003       gcctaAGCTAA
+r004                     ATAGCT..............TCAGC
-r003                            ttagctTAGGC
-r001/2                                        CAGCGCCAT
\end{verbatim}
\end{framed}
The corresponding SAM format is:
\begin{framed}\small
\begin{verbatim}
@HD VN:1.3 SO:coordinate
@SQ SN:ref LN:45
r001 163 ref  7 30 8M2I4M1D3M = 37  39 TTAGATAAAGGATACTG *
r002   0 ref  9 30 3S6M1P1I4M *  0   0 AAAAGATAAGGATA    *
r003   0 ref  9 30 5H6M       *  0   0 AGCTAA   *   NM:i:1
r004   0 ref 16 30 6M14N5M    *  0   0 ATAGCTTCAGC       *
r003  16 ref 29 30 6H5M       *  0   0 TAGGC    *   NM:i:0
r001  83 ref 37 30 9M         =  7 -39 CAGCGCCAT         *
\end{verbatim}
\end{framed}

\subsection{Terminologies and Concepts}

\begin{description}
\item[Template] A DNA/RNA sequence part of which is sequenced on a
  sequencing machine or assembled from raw sequences.
\item[Fragment] A contiguous (sub)sequence on a template which is
  sequenced or assembled. For sequencing data, fragments are indexed by
  the order in which they are sequenced. For fragments of an assembled
  sequence, they are indexed by the order of the leftmost coordinate on
  the assembled sequence.
\item[Read] A raw sequence that comes off a sequencing machine. A read
  may consist of multiple fragments.
\item[1-based coordinate system] A coordinate system where the first
  base of a sequence is one. In this coordinate system, a region is
  specified by a closed interval. For example, the region between the 3rd
  and the 7th bases inclusive is $[3,7]$. The SAM and GFF formats are
  using the 1-based coordinate system.
\item[0-based coordinate system] A coordinate system where the first
  base of a sequence is zero. In this coordinate system, a region is
  specified by a half-close-half-open interval. For example, the region
  between the 3rd and the 7th bases inclusive is $[2,7)$. The BED,
  Wiggle and PSL formats are using the 0-based coordinate system.
\item[Phred scale] Given a probability $0<p\le 1$, the phred scale of $p$
  equals $-10\log_{10}p$, rounded to the closest integer.
\end{description}

\subsection{The header section}
Each header line begins with character `{\tt @}' followed by a
two-letter record type code. In the header, each line is TAB-delimited
and each data field follows a format `{\tt TAG:VALUE}' where {\tt TAG}
is a two-letter string that defines the content and the format of {\tt
  VALUE}. Each header line should match:\\ {\tt
  /\char94@[A-Z][A-Z](\char92t[A-Z][A-Z]:[ -\char126])+\$/}.

The following table give the defined record types and tags. Tags with
`*' are required when the record type is present.
\begin{center}
\small
\begin{longtable}{|l|l|p{13.5cm}|}
  \cline{1-3}
  \multicolumn{2}{|l|}{\bf Tag} & {\bf Description} \\
  \cline{1-3}
  \multicolumn{2}{|l}{\tt @HD} & The header line. The first line if present. \\\cline{2-3}
  & {\tt VN}* & Format version. \emph{Accepted format}: {\tt /\char94[0-9]+\char92.[0-9]+\$/}.\\\cline{2-3}
  & {\tt SO} & Sorting order. \emph{Valid values}: {\tt unknown} (default), {\tt unsorted}, {\tt queryname} and {\tt coordinate}. \\\cline{1-3}
  \multicolumn{2}{|l}{\tt @SQ} & Reference sequence dictionary. The order of {\tt @SQ} lines defines the alignment sorting order.\\\cline{2-3}
  & {\tt SN}* & Reference sequence name. Each {\tt @SQ} line must have a unique {\tt SN} tag. The value of this
  field is used in the
  alignment records. \\\cline{2-3}
  & {\tt LN}* & Reference sequence length. \emph{Range}: {\tt [1,2$^{29}$-1]}\\\cline{2-3}
  & {\tt AS} & Genome assembly identifier. \\\cline{2-3}
  & {\tt M5} & MD5 checksum of the sequence in the uppercase, with gaps and spaces removed.\\\cline{2-3}
  & {\tt SP} & Species.\\\cline{2-3}
  & {\tt UR} & URI of the sequence.\\\cline{1-3}
  \multicolumn{2}{|l}{\tt @RG} & Read group. Unordered multiple lines are allowed.\\\cline{2-3}
  & {\tt ID}* & Read group identifier. Each {\tt @RG} line must have a unique {\tt ID}. The value of {\tt ID}
  is used in the RG tags of alignment records. \\\cline{2-3}
  & {\tt CN} & Name of sequencing center producing the read.\\\cline{2-3}
  & {\tt DS} & Description.\\\cline{2-3}
  & {\tt DT} & Date the run was produced (ISO8601 date or date/time).\\\cline{2-3}
  & {\tt LB} & Library.\\\cline{2-3}
  & {\tt PI} & Predicted median insert size.\\\cline{2-3}
  & {\tt PL} & Platform/technology used to produce the read. \emph{Valid values}:
  {\tt ILLUMINA}, {\tt SOLID}, {\tt LS454}, {\tt HELICOS} and {\tt PACBIO}.\\\cline{2-3}
  & {\tt PU} & Platform unit (e.g. lane for Illumina or slide for SOLiD). Unique identifier.\\\cline{2-3}
  & {\tt SM} & Sample. Use pool name where a pool is being sequenced.\\\cline{1-3}\pagebreak\hline
  \multicolumn{2}{|l}{\tt @PG} & Program. \\\cline{2-3}
  & {\tt ID}* & Program name. Each {\tt @PG} line must have a unique {\tt ID}. The value of {\tt ID} is used in the alignment {\tt PG} tag and {\tt PP} tags of other {\tt @PG} lines.\\\cline{2-3}
  & {\tt CL} & Command line \\\cline{2-3}
  & {\tt PP} & Previous {\tt @PG-ID}. Must match another {\tt @PG} header's {\tt ID} tag. {\tt @PG} records may be chained using {\tt PP} tag, with the first record in the chain having no {\tt PP} tag. This chain defines the order of programs that have been applied to the alignment. \\\cline{2-3}
  & {\tt VN} & Program version \\\cline{1-3}
  \multicolumn{2}{|l}{\tt @CO} & One-line text comment. Unordered multiple lines are allowed.\\
  \cline{1-3}
\end{longtable}
\end{center}

\subsection{The alignment section: mandatory fields}
Each alignment line has 11 mandatory fields. These fields always appear
in the same order and must be present, but their values can be `0' or
`*' (depending on the field) if the corresponding information is
unavailable. The following table gives an overview of the mandatory
fields in the SAM format:
\begin{center}
\small
\begin{tabular}{rllll}
  \hline
  {\bf Col} & {\bf Field} & {\bf Type} & {\bf Regexp/Range} & {\bf Brief description} \\
  \hline
  1 & {\sf QNAME} & String & {\tt [!-?A-\char126]+} & Query template NAME\\
  2 & {\sf FLAG} & Int/Str & {\tt [0,2$^{16}$-1]}/{\tt \char92*|[pPuUrR12sfd]+} & bitwise FLAG \\
  3 & {\sf RNAME} & String & {\tt \char92*|[!-)+-\char60\char62-\char126][!-\char126]*} & Reference sequence NAME\\
  4 & {\sf POS} & Int & {\tt [0,2$^{29}$-1]} & 1-based leftmost mapping POSition \\
  5 & {\sf MAPQ} & Int & {\tt [0,2$^8$-1]} & MAPping Quality \\
  6 & {\sf CIGAR} & String & {\tt \char92*|([0-9]+[MIDNSHPX=])+} & CIGAR string \\
  7 & {\sf RNEXT} & String & {\tt \char92*|=|[!-)+-\char60\char62-\char126][!-\char126]*} & Ref. name of the mate/next fragment\\
  8 & {\sf PNEXT} & Int & {\tt [0,2$^{29}$-1]} & Position of the mate/next fragment \\
  9 & {\sf TLEN} & Int & {\tt [-2$^{29}$+1,2$^{29}$-1]} & observed Template LENgth \\
  10 & {\sf SEQ} & String & {\tt \char92*|[A-Za-z=.]+} & fragment SEQuence\\
  11 & {\sf QUAL} & String & {\tt [!-\char126]+} & ASCII of base QUALity+33 \\
  \hline
\end{tabular}
\end{center}

\begin{enumerate}
\item {\sf QNAME}: Query template NAME. Each template has a unique name.
\item {\sf FLAG}: bitwise FLAG. Each bit is explained in the following
  table (`*' means no bits are set):
  \begin{center}\small
  \begin{tabular}{rcl}
  \hline
  Bit & Char & Description\\
  \hline
  0x1 & p & template having multiple fragments in sequencing \\
  0x2 & P & each fragment properly aligned according to the aligner \\
  0x4 & u & fragment unmapped \\
  0x8 & U & next fragment in the template unmapped \\
  0x10 & r & {\sf SEQ} being reverse complemented \\
  0x20 & R & {\sf SEQ} of the next fragment in the template being reversed \\
  0x40 & 1 & the first fragment in the template \\
  0x80 & 2 & the last fragment in the template \\
  0x100 & s & secondary alignment\\
  0x200 & f & not passing quality controls \\
  0x400 & d & PCR or optical duplicate \\
  \hline
  \end{tabular}
  \end{center}
  \begin{itemize}
  \item Bit 0x4 is the only reliable place to tell whether the fragment
    is unmapped. If 0x4 is set, no assumptions can be made about {\sf
      RNAME}, {\sf POS}, bits 0x2, 0x10 and 0x100 and the bit 0x20 of
    the next fragment in the template.
  \item If 0x40 and 0x80 are both set, the fragment is part of a linear
    template, but it is neither the first nor the last fragment. If both
    0x40 and 0x80 are unset, the index of the fragment in the template
    is unknown. This may happen for a non-linear template or the index
    is lost in data processing.
  \item Bit 0x100 marks the alignment not to be used in certain analyses
    when the tools in use are aware of this bit.
  \item If 0x1 is unset, no assumptions can be made about 0x2, 0x8,
    0x20, 0x40 and 0x80.
  \end{itemize}
\item {\sf RNAME}: Reference sequence NAME of the alignment. If {\tt
    @SQ} header lines are present, {\sf RNAME} (if not `*') must be
  present in one of the {\tt SQ-SN} tag. An unmapped fragment without
  coordinate has a `*' at this field. However, an unmapped fragment may
  also have an ordinary coordinate such that it can be placed at a
  desired position after sorting. If {\sf RNAME} is `*', no assumptions
  can be made about {\sf POS} and {\sf CIGAR}.
\item {\sf POS}: 1-based leftmost mapping POSition of the first matching
  base. The first base in a reference sequence has coordinate 1. {\sf
    POS} is set as 0 for an unmapped read without coordinate. If {\sf
    POS} is 0, no assumptions can be made about {\sf RNAME} and {\sf
    CIGAR}.
\item {\sf MAPQ}: MAPping Quality. It equals
  $-10\log_{10}\Pr\{\mbox{mapping position is wrong}\}$, rounded to the
  nearest integer. A value 255 indicates that the mapping quality is not
  available.
\item {\sf CIGAR}: CIGAR string. The CIGAR operations are given in the
  following table (set `*' if unavailable):
  \begin{center}\small
  \begin{tabular}{cl}
  \hline
  Op & Description\\
  \hline
  {\tt M} & alignment match (can be a sequence match or mismatch)\\
  {\tt I} & insertion to the reference \\
  {\tt D} & deletion from the reference \\
  {\tt N} & skipped region from the reference \\
  {\tt S} & soft clipping (clipped sequences present in {\sf SEQ})\\
  {\tt H} & hard clipping (clipped sequences NOT present in {\sf SEQ})\\
  {\tt P} & padding (silent deletion from padded reference)\\
  {\tt =} & sequence match \\
  {\tt X} & sequence mismatch \\
  \hline
  \end{tabular}
  \end{center}
  \begin{itemize}
  \item S/H can only be the first or the last operation.
  \end{itemize}
\item {\sf RNEXT}: Reference sequence name of the NEXT fragment in the
  template. For the last fragment, the next fragment is the first
  fragment in the template. If {\tt @SQ} header lines are present, {\sf
    RNEXT} (if not `*' or `=') must be present in one of the {\tt SQ-SN}
  tag. This field is set as `*' when the information is unavailable, and
  set as `=' if {\sf RNEXT} is identical {\sf RNAME}. If not `=' and the
  next fragment in the template has one primary mapping (see also bit
  0x100 in {\sf FLAG}), this field is identical to {\sf RNAME} of the
  next fragment.  If the next fragment has multiple primary mappings, no
  assumptions can be made about {\sf RNEXT} and {\sf PNEXT}.  If {\sf
    RNEXT} is `*', no assumptions can be made on {\sf PNEXT} and bit
  0x20.
\item {\sf PNEXT}: Position of the NEXT fragment in the template. Set as
  0 when the information is unavailable. This field equals {\sf POS} of
  the next fragment. If {\sf PNEXT} is 0, no assumptions can be made on
  {\sf RNEXT} and bit 0x20.
\item {\sf TLEN}: signed observed Template LENgth. If all fragments are
  mapped to the same reference, the unsigned observed template length
  equals the number of bases from the leftmost mapped base to the
  rightmost mapped base. The leftmost fragment has a plus sign and the
  rightmost has a minus sign. The sign of fragments in the middle is
  undefined. It is set as 0 for single-fragment template or when the
  information is unavailable.
\item {\sf SEQ}: fragment SEQuence. This field can be a `*' when the
  sequence is not stored. If not a `*', the length of the sequence must
  equal the sum of lengths of {\tt M/I/S/=/X} operations in {\sf CIGAR}.
  An `=' denotes the base is identical to the reference base. No
  assumptions can be made on the letter cases. Anything other than {\tt
    A/C/G/T/=} is regarded as ambiguous base {\tt N}.
\item {\sf QUAL}: ASCII of base QUALity plus 33. A base quality equals
  $-10\log_{10}\Pr\{\mbox{base is wrong}\}$. This field can be a `*'
  when quality is not stored. If not a `*', {\sf SEQ} is not a `*' and
  the length of the quality string must equal the length of {\sf SEQ}.
\end{enumerate}

\subsection{The alignment section: optional fields}
All optional fields are presented in the {\tt TAG:TYPE:VALUE} format
where {\tt TAG} is a two-character string that matches {\tt
  /[A-Z][A-Z0-9]/}, {\tt TYPE} is a casesensitive single letter which
defines the format of {\tt VALUE}:
\begin{center}\small
\begin{tabular}{cll}
\hline
{\bf Type} & {\bf Regexp matching {\tt VALUE}} & {\bf Descrption} \\
\hline
A & {\tt [!-\char126]} & Printable character \\
i & {\tt [-+]?[0-9]+} & Singed 32-bit integer \\
f & {\tt [-+]?[0-9]*\char92.?[0-9]+([eE][-+]?[0-9]+)?} & Single-precision floating number \\
Z & {\tt [\,\,\,!-\char126]+} & Printable string, including space\\
H & {\tt [0-9A-F]+} & Hex string, high nybble first \\
\hline
\end{tabular}
\end{center}
Each {\tt TAG} can only appear once in one alignment line.

{\flushleft Predefined tags are shown in the following table. You can
  freely add new tags, and if a new tag may be of general interest, you
  can email {\tt samtools-help@lists.sourceforge.net} to add the new tag
  to the specification. Note that tags started with `{\tt X}', `{\tt Y}'
  and `{\tt Z}' are reserved for local use and will not be formally
  defined in any future version of this specification.}
\begin{center}\small
\begin{tabular}{ccp{12.5cm}}
  \hline
  {\bf Tag} & {\bf Type} & {\bf Description} \\
  \hline
  {\tt X?} & ? & Reserved fields for end users (together with {\tt Y?} and {\tt Z?}) \\
  {\tt AM} & i & The smallest template-independent mapping quality of fragments in the rest \\
  {\tt AS} & i & Alignment score  generated by aligner \\
  {\tt CM} & i & Edit distance between the color sequence and the color reference (see also {\tt NM})\\
  {\tt CQ} & Z & Color read quality on the original strand of the read. Same encoding as {\sf QUAL}.\\
  {\tt CS} & Z & Color read sequence on the original strand of the read. Of the same length as {\tt CQ}. \\
  {\tt E2} & Z & The 2nd most likely base calls. Of the same length as {\sf SEQ}. \\
  {\tt FI} & i & The index of fragment in the template.\\
  {\tt FS} & Z & Fragment suffix.\\
  {\tt LB} & Z & Library. Value to be consistent with the header {\tt RG-LB} tag if {\tt @RG} is present.\\
  {\tt H0} & i & Number of perfect hits\\
  {\tt H1} & i & Number of 1-difference hits (see also {\tt NM})\\
  {\tt H2} & i & Number of 2-difference hits \\
  {\tt HI} & i & Query hit index, indicating the alignment record is the i-th one stored in SAM\\
  {\tt IH} & i & Number of stored alignments in SAM that contains the query in the current record\\
  {\tt MD} & Z & String for mismatching positions [TODO: add descriptions]\\
  {\tt MQ} & i & Mapping quality of the mate/next fragment \\
  {\tt NH} & i & Number of reported alignments that contains the query in the current record\\
  {\tt NM} & i & Edit distance to the reference, including ambiguous bases but excluding clipping \\
  {\tt OQ} & Z & Original base quality (usually before recalibration). Same encoding as {\sf QUAL}.\\
  {\tt OP} & i & Original mapping position (usually before realignment) \\
  {\tt OC} & Z & Original CIGAR (usually before realignment) \\
  {\tt PG} & Z & Program. Value matches the header {\tt PG-ID} tag if {\tt @PG} is present. \\
  {\tt PQ} & i & Phred likelihood of the template, conditional on both the mapping being correct \\
  {\tt PU} & Z & Platform unit. Value to be consistent with the header {\tt RG-PU} tag if {\tt @RG} is present.\\
  {\tt Q2} & Z & Phred quality of the mate/next fragment. Same encoding as {\sf QUAL}.\\
  {\tt R2} & Z & Sequence of the mate/next fragment in the template. \\
  {\tt RG} & Z & Read group. Value matches the header {\tt RG-ID} tag if {\tt @RG} is present in the header. \\
  {\tt SM} & i & Template-independent mapping quality \\
  {\tt TC} & i & The number of fragments in the template.\\
  {\tt U2} & Z & Phred probility of the 2nd call being wrong conditional on the best being wrong. The same encoding as {\sf QUAL}. \\
  {\tt UQ} & i & Phred likelihood of the fragment, conditional on the mapping being correct \\
  \hline
\end{tabular}
\end{center}
The {\tt GS}, {\tt GC}, {\tt GQ}, {\tt CC}, {\tt CP}, {\tt MF}, {\tt S2}
and {\tt SQ} are reserved for backward compatibility.

\pagebreak

\section{The SAM Format Standards}

\begin{enumerate}
\item The header section
  \begin{enumerate}[label*=\arabic*]
  \item The {\tt @HD} line is present with the {\tt SO} tag specified.
  \item The {\tt @SQ} lines are present if reads have been mapped.
  \item The corresponding {\tt @RG} lines are defined if {\tt RG} tags
    appear in the alignment lines.
  \item The corresponding {\tt @PG} lines are defined if {\tt PG} tags
    appear in the alignment lines.
  \end{enumerate}
\item CIGAR
  \begin{enumerate}[label*=\arabic*]
  \item Adjacent CIGAR operations are different.
  \item If {\sf POS} plus the sum of lengths of {\tt M/=/X/D/N}
    operations in {\sf CIGAR} exceeds the length specified in the {\tt
      LN} field of the {\tt @SQ} header line (if exists) with an SN
    equal to {\sf RNAME}, the alignment is unmapped.
  \end{enumerate}
\item No alignments are assigned mapping quality 255.
\item Unmapped reads
  \begin{enumerate}[label*=\arabic*]
  \item For a unmapped paired-end or mate-pair read whose mate is
    mapped, the unmapped read has {\sf RNAME} and {\sf POS} identical to
    its mate.
  \item If all fragments in a template are unmapped, their {\sf RNAME}
    is set as `*' and {\sf POS} as 0.
  \end{enumerate}
\item Multiple mapping
  \begin{enumerate}[label*=\arabic*]
  \item At most one primary alignment for each fragment (controlled
    by the 0x100 bit). {\sf RNEXT} and {\sf PNEXT} point to the primary
    alignment of the next fragment.
  \item {\sf SEQ} and {\sf QUAL} of secondary alignments are set to `*'.
  \end{enumerate}
\item Chimeric alignment
  \begin{enumerate}[label*=\arabic*]
  \item There is no overlap between fragments of a read (few/no existing
    aligners follow this standard).
  \item {\sf RNEXT} and {\sf PNEXT} describes the relationship between
    chimera, where appropriate.
  \end{enumerate}  
\end{enumerate}

\end{document}
